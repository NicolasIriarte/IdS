\documentclass[
  11pt, % The default document font size, options: 10pt, 11pt, 12pt
  codirector, % Uncomment to add a codirector to the title page
]{charter}
\usepackage{enumitem}
\usepackage{pdflscape}
\usepackage{tikz}
%\usetikzlibrary{shapes,arrows}
%\usepackage{tikz}
\usetikzlibrary{positioning, arrows.meta, backgrounds, fit}
\usepackage{fontawesome5}
\usepackage{tikz,tkz-tab}
\usepackage{booktabs} % Para tablas más elegantes
\usetikzlibrary{positioning, arrows.meta, backgrounds, fit}

\usetikzlibrary{matrix,arrows, positioning,shadows,shadings,backgrounds, calc, shapes, tikzmark}

\usepackage{fmtcount}

\makeatletter
\newcommand{\mytwodigits}[1]{\two@digits{#1}}
\makeatother

\newcounter{reqCounter}
\setcounter{reqCounter}{0}


% Completar los siguintes Campos
\materia{Ingeniería de Software}
\bimestre{tercer bimestre}
\docentes{Alejandro Permingeat; Esteban	Volentini; Mariano Finochietto y Santiago Salamandri}
\titulo{Emulador de microprocesador Leon3 para desarrollo de software satelital y simuladores}
\posgrado{Carrera de Especialización en Sistemas Embebidos}
\autor{Ing. Iriarte Fernandez, Nicolás Ezequiel
  (NicolasIriarte95@gmail.com)}
\director{Esp. Lic. Horro, Nicolás Eduardo}
\pertenenciaDirector{INVAP.\@S.E.}
\codirector{}
\pertenenciaCoDirector{}
\cliente{Pinedo, Matías}
\empresaCliente{INVAP.\@SE}
\fechaINICIO{16 de noviembre de 2023}		%Fecha de entrega
\CODrequerimiento{NEMU-UC-}


\begin{document}

\maketitle
\tableofcontents

\newpage

\section*{Registros de cambios}
\label{sec:registro}


\begin{table}[ht]
	\label{tab:registro}
	\centering
	\begin{tabularx}{\linewidth}{@{}|c|X|c|@{}}
		\hline
		\rowcolor[HTML]{C0C0C0}
		Revisión & \multicolumn{1}{c|}{\cellcolor[HTML]{C0C0C0}Detalles de los cambios realizados} & Fecha      \\ \hline
		0      & Creación del documento                                 &\fechaInicioName \\ \hline
		\hline

	\end{tabularx}
	\label{sec:cierre}
\end{table}

\pagebreak


\section{Introducción a los casos de uso}
\label{sec:org60390fa}

El presente documento describe los casos de uso del emulador de microprocesador Leon3 para desarrollo de software satelital y simuladores. Se presentán tres casos de uso representativos al uso nominal del software propuesto:

\begin{enumerate}
\item Cargar software de vuelo.
\item Inspeccionar registros y memorias.
\item Validar programa introducido al emulador.
\end{enumerate}

%%%%%%%%%%%%%%%%%%%%%%%%%%%%%%%%%%%%%%%%%%%%%%%%%%

\stepcounter{reqCounter}
\begin{table}[h!]
	\caption{Caso de Uso: Cargar software de vuelo [\CODrequerimiento\mytwodigits{\value{reqCounter}}]}
	\centering
	\begin{tabular}{ | m{4.0cm} | m{10cm} | }
		\hline
		\rowcolor{gray!50} % Coloring the first row
		\textbf{Título} & \textbf{Descripción} \\ % \hhline{|=|=|}
		\textbf{1. Nombre} & \textbf{Cargar software de vuelo} \\
		1.1 código &\textbf{[\CODrequerimiento\mytwodigits{\value{reqCounter}}]} \\
		1.2 Breve descripción & Capacidad de carga de software de vuelo arbitrario. \\
		1.3 Actor principal & Desarrolladores de FSW y operadores. \\
		1.4 Disparadores & Llamada a funcion por API. \\ \hline
		\textbf{2. Flujo de eventos} &  \\
		2.1 Flujo básico &
		\begin{enumerate}
			\item Se inicia una instancia de emulación.
      \item Se llama a la funcion de API para la carga de binarios.
      \item El software dado es cargado en la memoria emulada y se actualizan los registros correspondientes (Como contadores de programa).
      \item La función retorna al usuario.
		\end{enumerate} \\
		2.2 Flujo alternativo &
		\begin{itemize}
			\item Ante un binario inválido, como podría ser por compilador inválido o arquitectura incorrecta, la función retornará un código de error.
			\item Ante un archivo no existente, la función retornará un código de error.
		\end{itemize} \\ \hline
		\textbf{3. Requerimientos especiales} & N/A. \\ \hline
		\textbf{4. Pre-Condiciones} & Se posee un binario para ejecutar dentro del emulador. \\ \hline
		\textbf{5. Post-Condiciones} & Contador de programa actualizado y programa cargado en memoria. \\ \hline
	\end{tabular}

\end{table}

\stepcounter{reqCounter}


%%%%%%%%%%%%%%%%%%%%%%%%%%%%%%%%%%%%%%%%%%%%%%%%%%
\begin{table}[h!]
	\caption{Caso de Uso: Inspeccionar registros y memorias [\CODrequerimiento\mytwodigits{\value{reqCounter}}]}
	\centering
	\begin{tabular}{ | m{4.0cm} | m{10cm} | }
		\hline
		\rowcolor{gray!50} % Coloring the first row
		\textbf{Título} & \textbf{Descripción} \\ % \hhline{|=|=|}
		\textbf{1. Nombre} & \textbf{Inspeccionar registros y memorias} \\
		1.1 código &\textbf{[\CODrequerimiento\mytwodigits{\value{reqCounter}}]} \\
		1.2 Breve descripción & Acceso a los registros y memorias emuladas, tanto para verificación de funcionamiento como para desarrollo de drivers. \\
		1.3 Actor principal & Desarrolladores de FSW y Desarrolladores de modelos. \\
		1.4 Disparadores & Llamada a funcion por API. \\ \hline
		\textbf{2. Flujo de eventos} &  \\
		2.1 Flujo básico &
		\begin{enumerate}
			\item Se inicia una instancia de emulación.
      \item Se llama a la funcion de API para la carga de binarios.
      \item Se llama a la funcion de API para volcar los registros a archivo CSV.
      \item Alternativamente se puede llamar a la funcion de volcar memoria, especificando la posición initial, el tamaño a volcar y el archivo de salida para su analisis.
		\end{enumerate} \\
		2.2 Flujo alternativo &
		\begin{itemize}
			\item Ante una posición de memoria o memoria invalida la función retornará un código de error.
		\end{itemize} \\ \hline
		\textbf{3. Requerimientos especiales} & N/A. \\ \hline
		\textbf{4. Pre-Condiciones} & Se posee un binario para ejecutar dentro del emulador. \\ \hline
		\textbf{5. Post-Condiciones} & Se tendrá una instancia del emulador con un binario cargado y un archivo en formato CSV con los datos solicitados. \\ \hline
	\end{tabular}

\end{table}

\stepcounter{reqCounter}

%%%%%%%%%%%%%%%%%%%%%%%%%%%%%%%%%%%%%%%%%%%%%%%%%%
\begin{table}[h!]
	\caption{Caso de Uso: Validar programa introducido al emulador [\CODrequerimiento\mytwodigits{\value{reqCounter}}]}
	\centering
	\begin{tabular}{ | m{4.0cm} | m{10cm} | }
		\hline
		\rowcolor{gray!50} % Coloring the first row
		\textbf{Título} & \textbf{Descripción} \\ % \hhline{|=|=|}
		\textbf{1. Nombre} & \textbf{Validar programa introducido al emulador} \\
		1.1 código &\textbf{[\CODrequerimiento\mytwodigits{\value{reqCounter}}]} \\
		1.2 Breve descripción & El emulador permitirá ejecutar y validar binarios producidos para el microprocesador. \\
		1.3 Actor principal & Desarrolladores de FSW y Desarrolladores de modelos. \\
		1.4 Disparadores & Llamada a funcion por API. \\ \hline
		\textbf{2. Flujo de eventos} &  \\
		2.1 Flujo básico &
		\begin{enumerate}
			\item Se inicia una instancia de emulación.
      \item Se llama a la funcion de API para la carga de binarios.
      \item Se analiza si el binario dado realiza las operaciones esperadas.
		\end{enumerate} \\
		2.2 Flujo alternativo & N/A. \\ \hline
		\textbf{3. Requerimientos especiales} & N/A. \\ \hline
		\textbf{4. Pre-Condiciones} & Se posee un binario para ejecutar dentro del emulador con un comportamiento conocido. \\ \hline
		\textbf{5. Post-Condiciones} & Se tendrá una instancia del emulador con un binario cargado. \\ \hline
	\end{tabular}

\end{table}

\stepcounter{reqCounter}


\end{document}
